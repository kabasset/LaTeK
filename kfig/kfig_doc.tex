\documentclass[12pt,a4paper]{article}
\usepackage[latin1]{inputenc}
\usepackage[frenchb]{babel}
\usepackage[T1]{fontenc}
\usepackage{graphicx}
\usepackage{amssymb}
\usepackage[left=2cm,right=2cm,top=2cm,bottom=2cm]{geometry}

\usepackage{kfig}

%%% doc_cmd
\usepackage{framed}
\usepackage{xcolor}
\usepackage{xspace}

%%% Logo
\providecommand*{\Kfig}{\rotatebox{-12}{K}\hskip-.45ex
\textsc{f\hskip-.2ex
i\hskip-.3ex
g}\xspace}

\providecommand*{\kmath}{\rotatebox{-13}{K}\hskip-.5ex
\textsc{m\hskip-.2ex
a\hskip-.4ex
t\hskip-.15ex
h}\xspace}

\providecommand*{\kmisc}{\rotatebox{-13}{K}\hskip-.5ex
\textsc{m\hskip-.1ex
i\hskip-.1ex
s\hskip-.1ex
c}\xspace}


%%% Remark
\newcommand{\rmq}[1][]{\noindent\textcolor{red}{$\blacktriangleright\;$#1}}

%%% Simple code
\newenvironment{kcode}{\nobreak%
%\begin{framed}
\begin{center}
%\centering
%\smallskip
}{
%\normalcolor
\end{center}
%\end{framed}
\smallskip
}

%%% Code demonstration with captions
\newenvironment{kdemo}[3][r@{:\quad}]{\nobreak%
\begin{kcode}\nobreak
\renewcommand{\arraystretch}{1.5}\nobreak
\begin{tabular}{#1#2@{$\quad\leadsto\quad$}#3}
}{
\end{tabular}
\end{kcode}
}

%%% Code demonstration without captions
\newenvironment{kdemo*}[2]{\nobreak%
\begin{kdemo}[]{#1}{#2}
}{
\end{kdemo}
}

%%% colored verbatim
\newcommand{\emphverb}{\color{red}\verb}


\author{Antoine Basset}
\title{Documentation \Kfig}


\begin{document}
\maketitle

\begin{figure}[h]
\begin{kfig}{3}
\rule{\kimgw}{.5\kimgw} &
\rule{\kimgw}{.5\kimgw} &
\rule{\kimgw}{.5\kimgw}
\kskip
\rule{\kimgw}{.5\kimgw} &
\rule{\kimgw}{.5\kimgw} &
\rule{\kimgw}{.5\kimgw}
\end{kfig}
\caption{Une figure avec des images identiques}
\end{figure}

\begin{figure}[h]
\begin{kfig*}{3,1,2}
\rule{3\kimgh}{\kimgh} &
\rule{1\kimgh}{\kimgh} &
\rule{2\kimgh}{\kimgh}
\end{kfig*}
\kskip
\begin{kfig*}{1.5,4}
\rule{1.5\kimgh}{\kimgh} &
\rule{4\kimgh}{\kimgh}
\end{kfig*}
\caption{Une figure avec des images de tailles diff�rentes}
\end{figure}

\begin{figure}[h]
\begin{kfig*}{5,1.5}
\rule{5\kimgh}{\kimgh} &
Blabla:
bla, blabla bla.
Bla bla blabla, blablablabla!
Blaaa blabla, bla bla bla blabla...
\end{kfig*}
\caption{Une figure avec du texte}
\end{figure}

\pagebreak

\section{Images identiques}

L'environnement {\emphverb|kfig|} permet de cr�er des figures contenant plusieurs images remplissant enti�rement la largeur de la page (ou pas!).
Toutes les images auront la m�me largeur.
La syntaxe est la suivante:
\begin{kcode}
\verb|\begin{|%
{\emphverb|kfig|}%
\verb|}[|%
\emph{largeur de la figure}%
\verb|]{|%
\emph{nombre d'images}%
\verb|}|%
\end{kcode}
o�
\emph{largeur de la figure} est la largeur totale allou�e � la figure, \verb|\linewidth| par d�faut, \\
et
\emph{nombre d'images} est le nombre d'images par ligne.

L'environnement {\emphverb|kfig|} est bas� sur \verb|tabular|, de sorte qu'une figure de largeur 3~cm contenant 27~311 images par ligne, et deux lignes, serait cod�e ainsi:
\begin{kcode}
\begin{minipage}{.8\linewidth}
\verb|\begin{|%
{\emphverb|kfig|}%
\verb|}[3cm]{27311}|%

\begin{tabular}{l@{\texttt{~\&~}}l@{\texttt{~\&~}}l@{\texttt{~\&~}}l}
\emph{premi�re image} &
\emph{image 2} &
... &
\emph{image 27~311} \quad {\emphverb|\kskip|} \\
\emph{image 27~312} &
\emph{image 27~313} &
... &
\emph{derni�re image}
\end{tabular}

\verb|\end{|%
{\emphverb|kfig|}%
\verb|}|%
\end{minipage}
\end{kcode}
o� {\emphverb|\kskip|} permet de cr�er un espace vertical �gal aux espaces horizontaux entre les images.
Ces espacements sont contr�l�s par la longueur {\emphverb|\kimgsep|}.

Typiquement, \emph{premi�re image}, \emph{image 2}... sont des \verb|\includegraphics|.
Dans ce cas, il est inutile de pr�ciser la taille de l'image, qui sera ajust�e automatiquement � la largeur de la colonne {\emphverb|\kimgw|} (=\verb|\linewidth| soit dit en passant).


\section{Images de tailles diff�rentes}

Afin de cr�er des figures contenant plusieurs images de tailles diff�rentes, \Kfig d�finit l'environnement {\emphverb|kfig*|}, utilis� comme suit:
\begin{kcode}
\verb|\begin{|%
{\emphverb|kfig*|}%
\verb|}[|%
\emph{largeur de la figure}%
\verb|]{|%
\emph{liste des rapports largeur/hauteur des images}%
\verb|}|%
\end{kcode}
Ici, les images auront des largeurs diff�rentes, mais une hauteur commune, {\emphverb|\kimgh|}.

Ainsi, pour cr�er une figure contenant deux images, dont la premi�re a un format carr�, tandis que la seconde a une largeur 27~311 fois plus grande que sa hauteur, on �crira:
\begin{kcode}
\begin{minipage}{.8\linewidth}
\verb|\begin{|%
{\emphverb|kfig*|}%
\verb|}{1,27311}|%

\begin{tabular}{l@{\texttt{~\&~}}l}
\emph{premi�re image} &
\emph{seconde image}
\end{tabular}

\verb|\end{|%
{\emphverb|kfig*|}%
\verb|}|%
\end{minipage}
\end{kcode}

De nouveau, si vous utilisez \verb|\includegraphics| pour ins�rer vos images, la hauteur commune {\emphverb|\kimgh|} est impos�e par d�faut.

\rmq{
Le nombre de colonnes est limit� � 8 pour des raisons techniques, ce qui devrait �tre suffisant, sauf si vous �tes plus fou que moi.
Dans ce cas, d�brouillez-vous tout seul...
}


\section{Texte}

Si vous n'�tes pas rassasi�, et souhaitez remplacer une image par du texte, comme dans le dernier exemple... faites-le!
Rien ne vous emp�che d'ins�rer autre chose qu'une image.
Les colonnes des environnements {\emphverb|kfig|} et {\emphverb|kfig*|} sont de type \verb|p|.


\section{Exemples}

Pour l'inspiration, voici le code correspondant aux trois exemples introductifs.

\begin{verbatim}
\begin{figure}
\begin{kfig}{3}
  \rule{\kimgw}{.5\kimgw} &
  \rule{\kimgw}{.5\kimgw} &
  \rule{\kimgw}{.5\kimgw}
  \kskip
  \rule{\kimgw}{.5\kimgw} &
  \rule{\kimgw}{.5\kimgw} &
  \rule{\kimgw}{.5\kimgw}
\end{kfig}
\caption{Une figure avec des images identiques}
\end{figure}

\begin{figure}
\begin{kfig*}{3,1,2}
  \rule{3\kimgh}{\kimgh} &
  \rule{1\kimgh}{\kimgh} &
  \rule{2\kimgh}{\kimgh}
\end{kfig*}
\kskip
\begin{kfig*}{1.5,4}
  \rule{1.5\kimgh}{\kimgh} &
  \rule{4\kimgh}{\kimgh}
\end{kfig*}
\caption{Une figure avec des images de tailles diff�rentes}
\end{figure}

\begin{figure}
\begin{kfig*}{5,1.5}
  \rule{5\kimgh}{\kimgh} &
  Blabla:
  bla, blabla bla.
  Bla bla blabla, blablablabla!
  Blaaa blabla, bla bla bla blabla...
\end{kfig*}
\caption{Une figure avec du texte}
\end{figure}
\end{verbatim}


\end{document}