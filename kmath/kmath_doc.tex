\documentclass[10pt,a4paper]{article}
\usepackage[latin1]{inputenc}
\usepackage[english]{babel}
\usepackage{hyperref}
%\usepackage[left=2cm,right=2cm,top=2cm,bottom=2cm]{geometry}
\usepackage{kmath}

\clubpenalty=9999
\widowpenalty=9999

\usepackage{parskip}
\usepackage{enumitem}
\setlist{nosep}
\usepackage{cleveref} %TODO caps

%TODO greek packages
%\usepackage[greek,english]{babel}
\usepackage{textalpha} % nice
%\usepackage{upgreek}
%\usepackage{kpfonts}
%\usepackage{alphabeta}

%\knesteddelim

\usepackage{xspace}

%%% Logo
\providecommand*{\Kfig}{\rotatebox{-12}{K}\hskip-.45ex
\textsc{f\hskip-.2ex
i\hskip-.3ex
g}\xspace}

\providecommand*{\kmath}{\rotatebox{-13}{K}\hskip-.5ex
\textsc{m\hskip-.2ex
a\hskip-.4ex
t\hskip-.15ex
h}\xspace}

\providecommand*{\kmisc}{\rotatebox{-13}{K}\hskip-.5ex
\textsc{m\hskip-.1ex
i\hskip-.1ex
s\hskip-.1ex
c}\xspace}


%%% Remark
\newcommand{\rmq}[1][]{\noindent\textcolor{red}{$\blacktriangleright\;$#1}}

%%% Simple code
\newenvironment{kcode}{\nobreak%
%\begin{framed}
\begin{center}
%\centering
%\smallskip
}{
%\normalcolor
\end{center}
%\end{framed}
\smallskip
}

%%% Code demonstration with captions
\newenvironment{kdemo}[3][r@{:\quad}]{\nobreak%
\begin{kcode}\nobreak
\renewcommand{\arraystretch}{1.5}\nobreak
\begin{tabular}{#1#2@{$\quad\leadsto\quad$}#3}
}{
\end{tabular}
\end{kcode}
}

%%% Code demonstration without captions
\newenvironment{kdemo*}[2]{\nobreak%
\begin{kdemo}[]{#1}{#2}
}{
\end{kdemo}
}

%%% colored verbatim
\newcommand{\emphverb}{\color{red}\verb}

%%% Title
\title{The \kmath package}% \\\large [\kmisc distribution]}
\author{Antoine Basset}
\date{Version 0.9}

\begin{document}
\sloppy
\maketitle


%%%%%%%%%%%%%%%%
%%% ABSTRACT %%%
%%%%%%%%%%%%%%%%
\vfill
\begin{abstract}
\kmath provides commands to
1) save time when writing mathematical papers and
2) correctly typeset mathematical expressions, which are neither part of the standard {\LaTeX} math commands, nor yet implemented by the American Mathematical Society (\AmS).
\end{abstract}
\vfill


%%%%%%%%%%%%%
%%% INTRO %%%
%%%%%%%%%%%%%
\section*{Introduction}

\kmath defines a few commands to ease the writing of {\LaTeX} documents containing mathematical content:
\begin{description}
\item[nice and smart delimiters] (\cref{sec-delim}),
which adapt to their content size;
\item[classical notation] (\cref{sec-notation})
such as constants (e.g., $\kexp*(-\kiunit\kpi)$), bold sets (e.g., \kR*+), and other operators (e.g., $\kargmin_{a\in A} f(a)$);
\item[math macro definition] (\cref{ssec-kmacro}),
analog to {\LaTeX} \verb|\newcommand|,
but the defined macro can be used in text-mode and math-mode;
\item[function definition] (\cref{ssec-kfunc}),
a macro which may (or not) take arguments in parentheses, as well as sub- and superscripts before the argument%
{\newkfunc{\myfunc}{f}%
,e.g., \myfunc[i][2](\dfrac{1}{x}).
}
\end{description}

While all the defined commands work in math-mode (between \verb|$|'s, in equations, etc.), some of them are very often utilized in sentences;
these commands can be used directly in text-mode, as stated later in this document.

Usage of \kmath is extremely basic:
\begin{kcode}
\verb|\usepackage[|%
\textit{options}%
\verb|]{|%
{\emphverb|kmath|}%
\verb|}|
\end{kcode}
where \textit{options} are described hereafter and summarized in \cref{sec-options}.

\vfill
\pagebreak
\tableofcontents
\pagebreak


%%%%%%%%%%%%%%
%%% DELIMS %%%
%%%%%%%%%%%%%%
\section{Delimiters (parentheses \& co.)}
\label{sec-delim}

\subsection{Problem statement}
\label{ssec-delim-bug}

{\LaTeX} provides a set of delimiters
(produced with commands \verb|\left| and \verb|\right|)
which adapt to their content size, as shown bellow.
\begin{kdemo*}{c}{c}
\phantom{\tt left}\verb|(\frac{1}{x})|\phantom{\tt right} & $(\dfrac{1}{x})$ \\[2ex]
\verb|\left(\frac{1}{x}\right)| & $\left(\dfrac{1}{x}\right)$ \\
\end{kdemo*}

However, the spacing is not consistent with simple parentheses.
For example, compare the following:\nobreak
\begin{kdemo*}{l}{l}
\verb|\sin(x)| & $\sin(x)$ \\
\verb|\sin\left(x\right)| & $\sin\left(x\right)$ \\
\end{kdemo*}
As you see, spurious space has been added between ``$\sin$'' and ``$($''.
The well-known package \verb|mleftright| was intended to correct this issue, but some improvements can still be done.
Relying on \verb|mleftright|, \kmath provides shortcuts to avoid typing \verb|\mleft| and \verb|\mright| all the document long.


\subsection{Predefined delimiter pairs}
\label{ssec-delim-def}

Classical pairs of delimiters are obtained with predefined commands:
\begin{kdemo*}{l}{c}
{\emphverb|\kparen|}\verb|{x^2}| & $\kparen{x^2}$ \\
{\emphverb|\kbrace|}\verb|{x^2}| & $\kbrace{x^2}$ \\
{\emphverb|\kbracket|}\verb|{x^2}| & $\kbracket{x^2}$ \\
{\emphverb|\kangle|}\verb|{x^2}| & $\kangle{x^2}$ \\
{\emphverb|\kfloor|}\verb|{x^2}| & $\kfloor{x^2}$ \\
{\emphverb|\kceil|}\verb|{x^2}| & $\kceil{x^2}$ \\
%{\emphverb|\klcorner|}\verb|{x^2}| & $\klcorner{x^2}$ \\ %TODO bug
%{\emphverb|\kucorner|}\verb|{x^2}| & $\kucorner{x^2}$ \\ %TODO bug
{\emphverb|\kvbar|}\verb|{x^2}| & $\kvbar{x^2}$ \\
{\emphverb|\kvvbar|}\verb|{x^2}| & $\kvvbar{x^2}$ \\
\end{kdemo*}
If empty, the delimiters will not be displayed, that is \verb|\kparen{}| produces nothing!

\rmq[Advice --]{
Use the \kmath delimiters even in the simplest cases, as
1) they will ensure that every open delimiter is later closed
-- and vice versa --,
2) the inner expression may become more complex in the next revision of your document,
and
3) you can easily change the type of delimiters when reviewing the document.
}

\pagebreak %TODO
In the case of nested delimiters, it is often preferred to have them grow gradually.
This can be activated by command {\emphverb|\knesteddelim|}, and deactivated with {\emphverb|\kdefaultdelim|}.
\begin{kdemo*}{@{}l}{cc@{}}
\multicolumn{1}{c}{} &
Default style & Nested style \\
\hline
\verb|\kparen{\kparen{|...\verb|\kparen{x}|...\verb|}}| &
$\kparen{\kparen{\kparen{\kparen{\kparen{\kparen{x}}}}}}$ &
\knesteddelim
\rule{0pt}{6ex}$\kparen{\kparen{\kparen{\kparen{\kparen{\kparen{x}}}}}}$ \\
\end{kdemo*}

\subsection{Your own delimiters}
\label{ssec-delim-own}

If the proposed set of delimiters is not enough, you can still define your own ones using the {\emphverb|\kdelim|} command.
Here are a few examples:
\begin{kdemo*}{l}{l}
{\emphverb|\kdelim(|}\verb|{0,1}|{\emphverb|]|} &
$\kdelim( {0,1} ]$ \\[1ex]
{\emphverb|\kdelim.|}\verb|{\frac{\partial f}{\partial x}}|{\emphverb+|+}\verb|_0| &
$\kdelim. {\dfrac{\partial f}{\partial x}} |_0$ \\
\end{kdemo*}

Obviously, the first example is an interval, which is a very classical use case...
Which is why we provide dedicated macros in the next subsection!

\subsection{Intervals}
\label{ssec-interv}

The four types of intervals are defined.
The ``open delimiter'' is an outward bracket ($]$) by default,
but a parenthesis can be used instead with package option {\emphverb|intervparen|}.
Also, the comma delimiter can be changed to semicolon (;) with option {\emphverb|intervsemicolon|}.
\begin{kdemo*}{l}{cc}
\multicolumn{1}{c}{} & Bracket & Parenthesis \\
\hline
{\emphverb|\kintervcc|}\verb|{a}{b}| &
\kintervcc{a}{b} & \kintervcc{a}{b} \\
{\emphverb|\kintervoo|}\verb|{a}{b}| &
\kintervoo{a}{b} & $\kparen{a,b}$ \\
{\emphverb|\kintervco|}\verb|{a}{b}| &
\kintervco{a}{b} & $\kdelim[{a,b})$ \\
{\emphverb|\kintervoc|}\verb|{a}{b}| &
\kintervoc{a}{b} & $\kdelim({a,b}]$ \\
\end{kdemo*}
As you probably understood, {\emphverb|c|} (resp. {\emphverb|o|}) stands for ``closed'' (resp. ``open'').

\rmq{Intervals can be used both in math- and text-mode.}

%Similarly, we provide commands for integer intervals:
%\begin{kdemo*}{l}{c}
%{\emphverb|\kintcc|}\verb|{a}{b}| &
%\kintcc{a}{b} \\
%{\emphverb|\kintoo|}\verb|{a}{b}| &
%\kintoo{a}{b} \\
%{\emphverb|\kintoc|}\verb|{a}{b}| &
%\kintoc{a}{b} \\
%{\emphverb|\kintco|}\verb|{a}{b}| &
%\kintco{a}{b} \\
%\end{kdemo*}

\pagebreak %TODO
\subsection{Decorated delimiters}
\label{ssec-cond}

Relying on \kmath delimiters, commands are provided to quickly typeset indexed series, indexed family, conditional set, and conditional probability:
\begin{kdemo}{l}{l}
Indexed series &
{\emphverb|\kseries|}\verb|{u_n}{n>0}| & $\kseries{u_n}{n>0}$ \\
Indexed family &
{\emphverb|\kfamily|}\verb|{u_n}{n>0}| & $\kfamily{u_n}{n>0}$ \\
Conditional set &
{\emphverb|\kset|}\verb|{u_n}{n>0}| & $\kset{u_n}{n>0}$ \\
Conditional probability &
{\emphverb|\kproba|}\verb|{A}{B}| & $\kproba{A}{B}$ \\
\end{kdemo}

The middle delimiter of the set can be changed for colon ($\colon$) or comma (${}\mathbin{,}{}$) with package option {\emphverb|setcolon|} or {\emphverb|setcomma|}.

As for $\kproba{A}{B}$, ``$P$'' is the default value of an optional parameter, so that
{\emphverb|\kproba[|}\verb|p|{\emphverb|]|}\verb|{A}{B}|
yields \kproba[p]{A}{B}.

\rmq{All these commands can be used in math- or text-mode.}


\subsection{Breakable list}
\label{ssec-list}

Sometimes,~never-ending~lists~are~typed~in~text~mode,~e.g.,~%
$\mleft\{1, 1, 3, 3, 9, 9, 27, 27,..., 3^{\kfloor{\frac{N}{2}}}\mright\}$
is obviously too long to fit in the previous line.

Unfortunately, while variable size delimiters are preferable, \verb|\left| and \verb|\right| or \verb|\mleft| and \verb|\mright| forbid line breaks.

On the contrary,
{\emphverb|\kbrace*|} %\verb|{|\textit{content}\verb|}|
allows for typesetting breakable lists, so
\kbrace*{1,1,3,3,9,9,27,27,...,3^{\kfloor{\frac{N}{2}}}}
no longer overfills the hbox!

This example is merely obtained with:
\begin{kcode}
{\emphverb|\kbrace*|}%
\verb|{1,1,3,3,9,9,27,27,...,3^{\kfloor{\frac{N}{2}}}}|
\end{kcode}


If you want to avoid breaking the line between 3 and 9, add inner braces: \verb|{3,9}|.

Breakable versions of other delimiter pairs introduced in \cref{ssec-delim-def}
are defined as starred commands
{\emphverb|\kbrace*|},
{\emphverb|\kbracket*|},
{\emphverb|\kangle*|},
{\emphverb|\kfloor*|},
{\emphverb|\kceil*|},
{\emphverb|\kvbar*|} and
{\emphverb|\kvvbar*|}.

To define your own breakable delimiters, {\emphverb|\klist|} is analogous to {\emphverb|\kdelim|}.

\rmq{As opposed to unstarred versions of \cref{ssec-delim-def}, these commands \emph{are} available in text mode.}


%%%%%%%%%%%%%%%%%
%%% NOTATIONS %%%
%%%%%%%%%%%%%%%%%
\pagebreak %TODO
\section{Classical notation}
\label{sec-notation}

%%% Constants
\subsection{Constants}
\label{ssec-constants}

As opposed to previous standards (up to ISO~31), the latest mathematical style standard (ISO~80000-2) proposes to use upright font to typeset mathematical constants, such as the base of the natural logarithm $\kexp*$ or the imaginary unit $\kiunit$ (or $\kjunit$ in physics).
Naturally, \kmath provides appropriate macros, summarized below.
\begin{kdemo*}{l}{cc}
\multicolumn{1}{l}{} & ISO~31-11 & ISO~80000-2 \\
\hline
{\emphverb|\kexp*|} & $e$ & $\kexp*$ \\
{\emphverb|\kiunit|}\verb|,|{\emphverb|\kjunit|} & $i,j$ & $\kiunit,\kjunit$ \\
{\emphverb|\kpi|} & $\pi$ & $\kpi$ \\
\end{kdemo*}
However, if you prefer the old, italic, fashion, you can switch back to ISO~31 notation with package option {\emphverb|cstit|}.
%\begin{kcode}
%\verb|\usepackage[|{\emphverb|cstit|}\verb|]{kmath}|
%\end{kcode}
%In this case, why would you use a command while \verb|e| would be enough?
%Because if you change your mind, you just need to remove the option, and every constant will be typeset upright.

\rmq[Warning --]{
The upright pi symbol is not part of the standard {\LaTeX} math fonts, but is enabled by some packages: more on that in \cref{sec-dependencies}.
}

%%% Sets
\subsection{Bold sets}
\label{ssec-sets}

Classical bold sets are defined both in print and blackboard styles, as depicted below.
\begin{kdemo}{l}{cc}
\multicolumn{2}{c}{} & Print & Blackboard \\
\hline
%Primes &
%{\emphverb|\kP|} & \kP & $\mathbb{P}$ \\
Natural integers &
{\emphverb|\kN|} & \kN & $\mathbb{N}$ \\
Integers &
{\emphverb|\kZ|} & \kZ & $\mathbb{Z}$ \\
Decimal numbers &
{\emphverb|\kD|} & \kD & $\mathbb{D}$ \\
Rational numbers &
{\emphverb|\kQ|} & \kQ & $\mathbb{Q}$ \\
Algebraic numbers &
{\emphverb|\kA|} & \kA & $\mathbb{A}$ \\
Real numbers &
{\emphverb|\kR|} & \kR & $\mathbb{R}$ \\
Complex numbers &
{\emphverb|\kC|} & \kC & $\mathbb{C}$ \\
Quaternions &
{\emphverb|\kH|} & \kH & $\mathbb{H}$ \\
\end{kdemo}
If you prefer the blackboard bold style, use the {\emphverb|setbb|} package option:
\begin{kcode}
\verb|\usepackage[|{\emphverb|setbb|}\verb|]{kmath}|
\end{kcode}

\pagebreak %TODO
When mathematically sound, modifiers {\emphverb|*|}, {\emphverb|+|} and {\emphverb|-|} can be added to restrict the set:
\begin{kdemo*}{l}{l}
\verb|\kR| & \kR \\
\verb|\kR|{\emphverb|*|} & \kR* \\
\verb|\kR|{\emphverb|+|}\verb|,\kR|{\emphverb|-|} & \kR+, \kR- \\
\verb|\kR|{\emphverb|*+|}\verb|,\kR|{\emphverb|*-|} & \kR*+, \kR*- \\
\end{kdemo*}
The position of the star and sign can be changed to bottom or top with options {\emphverb|setstarb|} or {\emphverb|setsignt|}, respectively.
Those options are exclusive.
\begin{kdemo*}{l}{l}
\raggedright
\verb|\usepackage[|{\emphverb|setstarb|}\verb|]{kmath}| & $\kR_{*+}$ \\
\verb|\usepackage[|{\emphverb|setsignt|}\verb|]{kmath}| & $\kR^{*+}$ \\
\verb|\usepackage[setstarb,setsignt]{kmath}| & doesn't compile \\
\end{kdemo*}

%\subsubsection*{Your own bold set}

Yet, you can also define your own bold set with {\emphverb|\kboldset|}:
\begin{kdemo*}{l}{l}
{\emphverb|\kboldset|}\verb|{K}| & \kboldset{K} \\
{\emphverb|\kboldset*|}\verb|{K}| & \kboldset*{K} \\
{\emphverb|\kboldset+|}\verb|{K}| & \kboldset+{K} \\
{\emphverb|\kboldset*+|}\verb|{K}| & \kboldset*+{K} \\
\end{kdemo*}

\rmq{Bold sets can be used both in math- and text-mode.}

\subsection{Quantifiers}

Quantifiers ($\forall,\exists$) are considered as simple symbols in {\LaTeX}, so spacing is not properly handled.
\kmath provides quantifiers with two syntax styles:
followed by a comma (default behavior), or surrounded by parentheses as in logic.
To switch to the logic syntax, set package option {\emphverb|quantifparen|}.
\begin{kdemo*}{l}{rr}
\multicolumn{1}{c}{} &
\multicolumn{1}{c}{Standard} & \multicolumn{1}{c}{Logic} \\
\hline
{\emphverb|\kforall|}\verb|[x\in\kR] P| &
$\kforall[x\in\kR] P$ & $\mathop{}\kparen{\kforall x\in\kR}\mathop{} P$ \\
{\emphverb|\kexists|}\verb|[x\in\kR] P| &
$\kexists[x\in\kR] P$ & $\mathop{}\kparen{\kexists x\in\kR}\mathop{} P$ \\
{\emphverb|\knexists|}\verb|[x\in\kR] P| &
$\knexists[x\in\kR] P$ & $\mathop{}\kparen{\knexists x\in\kR}\mathop{} P$ \\
{\emphverb|\kexistsone|}\verb|[x\in\kR] P| &
$\kexistsone[x\in\kR] P$ & $\mathop{}\kparen{\kexistsone x\in\kR}\mathop{} P$ \\
\end{kdemo*}

The argument of those functions is optional; \verb|\kforall x| merely produces $\kforall x$ without comma or parentheses.

%%% Functions
\subsection{Operators and functions}

%The following math operators are provided, which do not need explanations...
\newcommand{\note}[1]{*${}^{#1}$}
\begin{kdemo}{l}{cl}
Transpose matrix &
{\emphverb|\ktranspose|}\verb|{A}| &
$\phantom{{}^\text{T}}\ktranspose{A}$ & \note1 \\
Adjoint matrix &
{\emphverb|\kadjoint|}\verb|{A}| &
$\phantom{{}^*}\kadjoint{A}$ \\
Inverse &
{\emphverb|\kinv|}\verb|{x}| &
$\;\kinv{x}$ \\
Kronecker delta &
{\emphverb|\kdelta|}\verb|_{ij}| &
$\phantom{{}_{ij}}\kdelta_{ij}$ & \note2 \\
Convolution operator &
\verb|f |{\emphverb|\kconvolve|}\verb| g| &
$f \kconvolve g$ \\
Composition operator &
\verb|f |{\emphverb|\kcompose|}\verb| g| &
$f \kcompose g$ \\
Nabla operator &
{\emphverb|\knabla|}\verb| f| &
$\knabla f$ & \note3 \\
Laplace operator &
{\emphverb|\klaplace|}\verb| f| &
$\klaplace f$ & \note3 \\
d'Alembert operator &
{\emphverb|\kdalembert|}\verb| f| &
$\kdalembert f$ \\
Real part &
{\emphverb|\kRe|}\verb| z| &
$\kRe{z}$ & \note4 \\
Imaginary part &
{\emphverb|\kIm|}\verb| z| &
$\kIm{z}$ & \note4\\
\end{kdemo}
\begin{itemize}
\item[\note1]
Package option {\emphverb|bourbakit|} yields the Bourbaki prescript notation for {\emphverb|\ktranspose|} (${}^\text{t}A$)
\item[\note2]
As \verb|\kpi|, {\emphverb|\kdelta|} will produce a roman (upright) symbol only if an adequate package is loaded before \kmath (see \cref{sec-dependencies}).
\item[\note3]
in order to be consistent with the d'Alembert operator,
\raisebox{\depth}{$\bigtriangledown$} and $\bigtriangleup$ can be used (instead of $\knabla$ and $\klaplace$) with package option {\emphverb|dalembert|}.
\item[\note4]
Old standard, Gothic versions of $\kRe$ and $\kIm$ ($\Re$ and $\Im$) are used if package option {\emphverb|reimgoth|} is passed.
\end{itemize}


To lighten writing, trigonometric functions are extended with two optional arguments:
the fist one, between brackets, corresponds to an exponent;
the second one, between parentheses, corresponds to the actual function argument.
\begin{kdemo*}{l}{l}
{\emphverb|\ksin|} & \ksin \\
{\emphverb|\ksin|}\verb|[2]| & \ksin[2] \\
{\emphverb|\ksin|}\verb|(x)| & \ksin(x) \\
{\emphverb|\ksin|}\verb|[2](x)| & \ksin[2](x)
\end{kdemo*}

Commands {\emphverb|\kcos|}, {\emphverb|\ktan|}, {\emphverb|\kcot|},
{\emphverb|\ksec|}, {\emphverb|\kcsc|},
{\emphverb|\ksinh|}, {\emphverb|\kcosh|}, {\emphverb|\ktanh|}, and {\emphverb|\kcoth|} are also provided.

\pagebreak %TODO
Exponential and logarithm functions are defined as follows:
\begin{kdemo*}{l}{l}
{\emphverb|\kexp|} & \kexp \\
{\emphverb|\kexp|}\verb|(x)| & \kexp(x) \\
{\emphverb|\kexp*|} & \kexp* \\
{\emphverb|\kexp*|}\verb|(x)| & \kexp*(x) \\
{\emphverb|\klog|} & \klog \\
{\emphverb|\klog|}\verb|[b]| & \klog[b] \\
{\emphverb|\klog|}\verb|(x)| & \klog(x) \\
{\emphverb|\klog|}\verb|[b](x)| & \klog[b](x) \\
{\emphverb|\kln|}\verb|(x)| & \kln(x) \\
{\emphverb|\klb|}\verb|(x)| & \klb(x)
\end{kdemo*}

Writing the full definition of a function is painful at best, so \kmath provides a nice command for that purpose, whose syntax is:
\begin{kcode}
{\emphverb|\kfuncdef|}%
\verb|[|%
\textit{pos}%
\verb|]{|%
\textit{function}%
\verb|}{|%
\textit{domain}%
\verb|}{|%
\textit{codomain}%
\verb|}[|%
\textit{argument}%
\verb|][|%
\textit{image}%
\verb|]|%
\end{kcode}
where \textit{pos} is a vertical alignment option, which can be either
{\emphverb|b|} (bottom, default),
{\emphverb|m|} (middle) or
{\emphverb|t|} (top).
\verb|b| was chosen as the default value so that the equation label is in front of the mapping definition, but that is a matter of taste.
The effect of \textit{pos} is demonstrated bellow.
\begin{kdemo*}{l}{lr}
{\emphverb|\kfuncdef|}%
\verb|{f}{E}{F}| &
$\kfuncdef[b]{f}{E}{F}$ & (1)\\[2ex]
{\emphverb|\kfuncdef|}%
\verb|{f}{E}{F}[x][y]| &
\begingroup\renewcommand{\arraystretch}{1}
$\kfuncdef[b]{f}{E}{F}[x][y]$
\endgroup & (1)\\[3ex]
{\emphverb|\kfuncdef[m]|}%
\verb|{f}{E}{F}[x][y]| &
\begingroup\renewcommand{\arraystretch}{1}
$\kfuncdef[m]{f}{E}{F}[x][y]$
\endgroup & (1)\\[3ex]
{\emphverb|\kfuncdef[t]|}%
\verb|{f}{E}{F}[x][y]| &
\begingroup\renewcommand{\arraystretch}{1}
$\kfuncdef[t]{f}{E}{F}[x][y]$
\endgroup & (1) \\
\end{kdemo*}

\subsection{arg min \textit{et alii}}

The following ``subscriptable'' operators are defined:
{\emphverb|\kargmin|},
{\emphverb|\kargmax|},
{\emphverb|\karginf|},
{\emphverb|\kargsup|},
{\emphverb|\kminimize|},
{\emphverb|\kmaximize|},
{\emphverb|\kminimise|}, and
{\emphverb|\kmaximise|}.
They have different looks in in-line and display styles.
\begin{kdemo*}{l}{cc}
\multicolumn{1}{c}{} & In-line & Display \\
\hline
{\emphverb|\kargmin|}\verb|_{x\in\kR} x^2| &
$\kargmin_{x\in\kR} x^2$ &
$\displaystyle\kargmin_{x\in\kR} x^2$ \\
\end{kdemo*}


%%%%%%%%%%%%%%
%%% MACROS %%%
%%%%%%%%%%%%%%
\pagebreak %TODO
\section{Your own mathematical macros}
\label{sec-macros}

\subsection{Math- and text-mode compatible macros}
\label{ssec-kmacro}

{\emphverb|\newkmacro|} is similar to \verb|\newcommand|, but the so-defined macro can be used both in math- and text-mode:
\begin{kcode}
{\emphverb|\newkmacro|}\verb|{|%
\textit{macro}%
\verb|}[|%
\textit{\#arguments}%
\verb|][|%
\textit{default}%
\verb|]{|%
\textit{definition}%
\verb|}|
\end{kcode}
where:
\begin{itemize}
\item \textit{macro}
is the macro name, which will be used later in the document, for example \verb|\foo|;
\item \textit{\#arguments} (optional)
is the number of macro arguments;
\item \textit{default} (optional)
is the default value of the first argument;
\item \textit{definition}
is the actual macro definition, which will replace \verb|\foo| later in the document.
\end{itemize}

In the next example, a macro \verb|\foo| is defined.
The macro takes one optional parameter, which is empty by default.
Each time \verb|\foo| appears in the document, it will be replaced by $\phi$, while \verb|\foo[bar]| will be replaced by $\phi_{\text{bar}}$, either in math-mode or text-mode.
\begin{kcode}
%\renewcommand{\arraystretch}{1.5}
\begin{tabular}{l}
{\emphverb|\newkmacro|}\verb|{\foo}[1][]{\phi_{\text{#1}}}| \\[-1ex]
$\vdots$ \\
\verb|\begin{document}| \\[-1ex]
$\vdots$ \\
\verb|Let \foo be the sum of \foo[bar] and \foo[qux]:| \\
\verb|$\foo = \foo[bar] + \foo[qux]$.| \\[-1ex]
$\vdots$ \\
\verb|\end{document}| \\
\\\hline\\
\newkmacro{\foo}[1][]{\phi_{\text{#1}}}%
Let \foo be the sum of \foo[bar] and \foo[qux]: $\foo=\foo[bar]+\foo[qux]$. \\
\end{tabular}
\end{kcode}

%\pagebreak %TODO
\subsection{Function-like macros}
\label{ssec-kfunc}

Command {\emphverb|\newkfunc|} is designed to handle mathematical functions.
The definition syntax is simple:
\begin{kcode}
{\emphverb|\newkfunc|}\verb|{|%
\textit{function}%
\verb|}{|%
\textit{definition}%
\verb|}|
\end{kcode}

For the sake of originality, let us introduce a function \verb|\foo| in preambule:
\begin{kcode}
\verb|\newkfunc{\foo}{\phi}|
\end{kcode}

\pagebreak %TODO
It can now be used in math- or text-mode in various ways...
\newkfunc{\foo}{\phi}
\begin{kdemo}{l}{l}
With no argument &
\verb|\foo| &
\foo \\
With an argument &
\verb|\foo(\frac{1}{x})| &
\foo(\dfrac{1}{x}) \\
With a subscript &
\verb|\foo[1]| &
\foo[1] \\
With a superscript &
\verb|\foo[][2]| &
\foo[][2]
\end{kdemo}
... which can be combined:
\begin{kdemo*}{l}{l}
\verb|\foo[1][2](\frac{1}{x})| &
\foo[1][2](\dfrac{1}{x}) \\
\end{kdemo*}

\rmq[Warning --]{
To ease writing and reading, the argument is typed between \emph{parentheses}, and not braces.
}


\pagebreak
%%%%%%%%%%%%%%%
%%% OPTIONS %%%
%%%%%%%%%%%%%%%
\section{Options summary}
\label{sec-options}

\begin{center}
\renewcommand{\arraystretch}{1.5}
\begin{tabular}{llc@{$\quad\to\quad$}c}
Option & Description & \multicolumn{2}{c}{Sample result} \\
\hline
{\emphverb|intervparen|} & Use parentheses in open intervals &
\kintervoo{\cdot}{\cdot} & $\kparen{\cdot,\cdot}$ \\
{\emphverb|intervsemicolon|} & Use semicolon in intervals &
\kintervcc{\cdot}{\cdot} & $\kbracket{\cdot;\cdot}$ \\
{\emphverb|setbb|} & Blackboard style bold sets &
\kR*+ & $\mathbb{R}^{*\phantom+}_+$ \\
{\emphverb|setstarb|} & Star of starred sets as subscript &
\kR*+ & $\kR_{*+}$ \\
{\emphverb|setsignt|} & Sign of signed sets as superscript &
\kR*+ & $\kR^{*+}$ \\
{\emphverb|setcolon|} & Colon as middle delimiter in sets &
$\kset{\cdot}{\cdot}$ & $\kbrace{\cdot\colon\cdot}$ \\
{\emphverb|setcomma|} & Comma as middle delimiter in sets &
$\kset{\cdot}{\cdot}$ & $\kbrace{\cdot\mathbin{,}\cdot}$ \\
{\emphverb|cstit|} & Italicized mathematical constants &
$\kexp*$ & $e$ \\
{\emphverb|quantifparen|} & Logic style quantifiers &
$\kforall[\cdot]\cdot$ & $\mathop{}\kparen{\kforall\cdot}\mathop{}\cdot$ \\
{\emphverb|bourbakit|} & Prescript transposition symbol &
$\ktranspose{A}$ & ${}^\text{t}A$ \\
{\emphverb|dalembert|} & Light Laplace and nabla operators &
$\Delta$ & $\bigtriangleup$ \\
{\emphverb|reimgoth|} & Gothic real and imaginary parts &
$\kRe$ & $\Re$ \\
\end{tabular}
\end{center}


%%%%%%%%%%%%%%
%%% INPUTS %%%
%%%%%%%%%%%%%%
\vfill %TODO
\section{Dependencies}
\label{sec-dependencies}

\kmath relies on the following packages, which are automatically loaded if needed:
\begin{itemize}
\item \verb|amsmath|
\item \verb|amssymb|
\item \verb|mleftright|
\item \verb|xparse|
\item \verb|xspace|
\end{itemize}

Optionally, if one of the following packages is loaded \emph{before} \kmath, the upright pi (resp. delta) character will be obtained with {\emphverb|\kpi|} (resp {\emphverb|\kdelta|}).
Otherwise, standard {\LaTeX} (italic) Greek letters will be substituted.
\begin{itemize}
\item \verb|babel|
with Greek language (e.g. \verb|\usepackage[greek,english]{babel}|)
\item \verb|textalpha|, which is used in this document
\item \verb|upgreek|
\item \verb|kpfonts|
\item \verb|mathdesign|
\item \verb|alphabeta|
\end{itemize}
\rmq[Warning --]{
Appearance of $\kpi$ and $\kdelta$ may vary depending on the loaded package and document font.
}

\end{document}